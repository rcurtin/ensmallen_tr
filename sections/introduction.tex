\section{Introduction}
\label{sec:introduction}

The problem of mathematical optimization is fundamental in the computational
sciences.  In short, this problem is expressed as
%
\begin{equation}
\operatorname{argmin}_x f(x).
\end{equation}

The ubiquity of this problem gives rise to the proliferation of mathematical
optimization toolkits, such as SciPy~\cite{2019arXiv190710121V},
opt++~\cite{meza1994opt++},
OR-Tools~\cite{ortools}, CVXOPT~\cite{vandenberghe2010cvxopt},
NLopt~\cite{johnson2014nlopt}, Ceres~\cite{ceres-solver},
and RBFOpt~\cite{costa2018rbfopt}.
Furthermore, in the field of machine learning, many
deep learning frameworks have integrated optimization
components.  Examples inclulde Theano~\cite{2016arXiv160502688},
TensorFlow~\cite{tensorflow2015-whitepaper}, PyTorch~\cite{NEURIPS2019_9015},
and Caffe~\cite{jia2014caffe}.

Mathematical optimization is generally quite computationally intensive.
For instance, the training of deep neural networks is dominated by
the optimization of the model parameters on the
data~\cite{krizhevsky2012imagenet, lauzon2012introduction}.  Similarly,
other popular machine learning algorithms such as logistic regression are also
expressed as and dominated by an optimization process~\cite{zhang2004solving,
manogaran2018health}.  Computational bottlenecks occur even in fields as
wide-ranging as rocket landing guidance systems~\cite{dueri2016customized},
motivating the development and implementation of specialized solvers.

The necessity of efficient and specializable function optimization motivated us
to implement the {\tt ensmallen} C++ library, originally as a part of the {\tt
mlpack} machine learning library~\cite{mlpack2018}.
Template metaprogramming techniques~\cite{abrahams2004c++,alexandrescu2001modern,
veldhuizen1998c++} are used to produce efficient code during compilation,
while simultaneously providing a friendly and intuitive interface
that matches the ease of use of popular optimization toolkits like SciPy,
OR-Tools, and MATLAB's {\tt fminsearch()} function~\cite{matlab_fminsearch}.

% taking advantage of special structures in the objective function,
%% CS: took out the above part, as it's unwieldy in the context
%% CS: and it's not clear what "special structures" is referring to.

% the 46 optimizers are as of 2.11.0
The library allows the user to define a function $f(x)$ to be optimized with
a wide variety of optimization techniques; at the time of this writing, 46
optimizers are available.  Many of these optimization techniques operate on
specific {\it classes} of objective functions---for instance, {\tt ensmallen}
contains many optimizers that work on differentiable functions, and so a user
must provide both $f(x)$ and $f'(x)$ in this case.
Through template metaprogramming, {\tt ensmallen} allows the user to select
an arbitrary type for $x$ (e.g., dense floating-point matrix, sparse integer matrix, etc.).
Users may also specify custom behavior during optimization via customizable {\it callbacks}.
The framework also facilitates the implementation of new optimization techniques.


We introduce the functionality and interface for {\tt ensmallen} in Section~\ref{sec:api}.
Section~\ref{sec:templated_optimize} shows how {\tt ensmallen} can optimize objective functions
$f(x)$ for various types of $x$ without any extra runtime overhead.
Section~\ref{sec:automatic} discusses the details of how {\tt ensmallen} is able
to automatically generate methods not provided by the user.
Section~\ref{sec:callbacks} describes {\tt ensmallen}'s callbacks.
We demonstrate the empirical efficiency of {\tt ensmallen} in
Section~\ref{sec:experiments} and conclude in Section~\ref{sec:conclusion}.

%% TODO: this will need to be refactored once the document is refactored
% In this paper, we describe the details of the template metaprogramming
% techniques and how we are able to simultaneously produce efficient code and a
% clean user interface.

This paper is a revised and expanded description of our initial overview
of {\tt ensmallen}~\cite{ensmallen2018}. 
It also provides a deep dive into the internals
of how the library works, which can be a very useful resource for anyone looking
to contribute to the library or get involved with its development.
