\section{Introduction}
\label{sec:introduction}

The problem of mathematical optimization is fundamental in the computational
sciences.  In short, this problem is expressed as
%
\begin{equation}
\operatorname{argmin}_x f(x).
\end{equation}

The ubiquity of this problem gives rise to the proliferation of mathematical
optimization toolkits, such as SciPy~\cite{2019arXiv190710121V},
opt++~\cite{meza1994opt++},
OR-Tools~\cite{ortools}, CVXOPT~\cite{vandenberghe2010cvxopt},
NLopt~\cite{johnson2014nlopt}, Ceres~\cite{ceres-solver},
and RBFOpt~\cite{costa2018rbfopt}.
Furthermore, in the field of machine learning, many
deep learning frameworks have integrated optimization
components.  Examples inclulde Theano~\cite{2016arXiv160502688},
TensorFlow~\cite{tensorflow2015-whitepaper}, PyTorch~\cite{NEURIPS2019_9015},
and Caffe~\cite{jia2014caffe}.

Function optimization is generally quite computationally intensive.
For instance, the training of deep neural networks is dominated by
the optimization of the model parameters on the
data~\cite{krizhevsky2012imagenet, lauzon2012introduction}.  Similarly,
other popular machine learning algorithms such as logistic regression are also
expressed as and dominated by an optimization process~\cite{zhang2004solving,
manogaran2018health}.  Computational bottlenecks occur even in fields as
wide-ranging as rocket landing guidance systems~\cite{dueri2016customized},
motivating the development and implementation of specialized solvers.

Many general-purpose optimization libraries do not make full use of the
hardware on which they are running.  Often, in order to make a clean and
easy-to-use API, general-purpose optimization libraries use design patterns that
implicitly make performance compromises.
\TODO{No clear link between ``making full use of hardware'' and ``design patterns''.
Suggest to delete the first sentence, as it's too general and doesn't link with anything.}
For instance, SciPy's {\tt minimize()}
method takes the function {\tt fun} to be optimized as a parameter;
This will mean that every time {\tt fun} is called (e.g., to compute the
value of $f(x)$ during the optimization routine), there will be overhead
involved---it will be slower than if the optimization routine were written {\it
specifically} for the function {\tt fun}.
\TODO{The point being made here is not clear. Very hand-wavy.
Where does the inefficiency come from? Why is there overhead?
Python has inherent overhead in handling of objects so there is no news here.
Each time {\tt fun} is referenced in Python code,
Python first needs to work out what is {\tt fun}.
In C++ there is no such overhead, as all types are known beforehand.
So in this limited sense ensmallen doesn't bring anything new to the table.
The ``Python vs C++'' argument is not novel,
so another angle of attack is required.
I suggest to focus on the relative breadth of the ensmallen framework,
as listed in Table~\ref{tab:comparison}.
It can be mentioned in passing that ``C++ also has an inherent advantage
in that all types are known beforehand thereby avoiding the inevitable
Python overhead''. However, this cannot be a central point.}

It is inefficiencies such as these that motivated us to implement the {\tt
ensmallen} library in C++, originally as a part of the {\tt mlpack} machine
learning library~\cite{mlpack2018}.  {\tt ensmallen} uses template
metaprogramming techniques to produce efficient code during compilation; through
these techniques, we are able to simultaneously provide a friendly and intuitive
interface that matches the ease of use of popular optimization toolkits like
SciPy, OR-Tools, and MATLAB's {\tt fminsearch()}
function~\cite{matlab_fminsearch}.
\TODO{I'm not convinced that providing the function to be optimized
as a template parameter brings in much efficiency gains over simply
providing a pointer to the function (ala NLopt).
Perhaps for trivial functions and/or processing only small vectors
this may make a bit of difference, but in general it's simply not worth it
due to added complexity and increased compilation time.
In many cases the bottleneck is matrix multiplication
(eg. code in Fig.~\ref{fig:lr_function}),
so specializing an optimizer via template argument isn't going to provide much speedups.
Template metaprogramming is, however, useful for determining which methods are available.}


% the 46 optimizers are as of 2.11.0
{\tt ensmallen} allows the user to define a function $f(x)$ to be optimized with
a wide variety of optimization techniques (at the time of this writing, 46
optimizers are available).  Many of these optimization techniques operate on
specific {\it classes} of objective functions---for instance, {\tt ensmallen}
contains many optimizers that work on differentiable functions, and so a user
must provide both $f(x)$ and $f'(x)$ in this case.
Through template metaprogramming, {\tt ensmallen} allows the user to select
an arbitrary type for $x$ (e.g., dense floating-point matrix, sparse integer matrix, etc.).
Users may also specify custom behavior via customizable {\it callbacks}.
It is also easy to implement new optimization techniques employing the framework provided by {\tt ensmallen}.

In this paper, we describe the details of these template metaprogramming
techniques and how we are able to simultaneously produce efficient code and a
clean user interface.  In Section~\ref{sec:api}, we introduce the functionality
and interface for {\tt ensmallen} with examples.
Section~\ref{sec:templated_optimize} is our first foray into template
metaprogramming, showing how {\tt ensmallen} can optimize objective functions
$f(x)$ for any type $x$ without any extra runtime overhead.
Section~\ref{sec:automatic} discusses the details of how {\tt ensmallen} is able
to automatically generate methods not provided by the user.
Section~\ref{sec:callbacks} describes {\tt ensmallen}'s callbacks.
We demonstrate the empirical efficiency of {\tt ensmallen} in
Section~\ref{sec:experiments} and conclude in Section~\ref{sec:conclusion}.

This paper is an expanded description of {\tt ensmallen}'s original
introduction~\cite{ensmallen2018, curtin2017generic}, and provides a deep dive
into the internals of how the library works.  This means it can be a very useful
resource for anyone looking to contribute to the library or get involved with
its development.
