\section{Automatically Inferring Missing Methods}
\label{sec:automatic}

In Section~\ref{sec:linreg_example}, we saw an example of how a user might implement {\tt
Evaluate()} and {\tt Gradient()} for the linear regression objective function
and use {\tt ensmallen} to find the minimum.
However, there is an inefficiency:
the objective function computation is defined as $f(\bm \theta) = \| \bm X \bm \theta - \bm y \|^2$,
and the gradient computation is defined as $f'(\bm \theta) = 2 \bm X^T (\bm X \bm \theta - \bm y)$.
There is a shared inner computation in $f(\bm \theta)$ and $f'(\bm \theta)$: the
term $(\bm X \bm \theta - \bm y)$.
If $f(\bm \theta)$ and $f'(\bm \theta)$ are implemented as separate functions,
there is no easy way to exploit this shared computation.

The differentiable optimizers in {\tt ensmallen} treat the given functions as oracular,
and do not know anything about the internal computations of the functions.
This inefficiency%
\footnote
  {It may be possible to use auto-differentiation to avoid this efficiency,
  or a programming language with introspection to operate directly on the
  abstract syntax tree~\cite{TODO} \TODO{citation} of the given objective and gradient
  computations to successfully share the computation.  However, at the time of
  this writing, we are not aware of any that do this.
  }
can apply to any optimization package that accepts an objective
function and its gradient as separate parameters,
such as SciPy, {\tt Optim.jl} and {\tt bfgsmin()} from Octave~\cite{TODO}\TODO{citation}.

To work around this issue, we use template metaprogramming techniques to allow
the user to provide {\it either} separate implementations of the objective
function and gradient, {\it or} a combined implementation that computes {\it
both} the objective function and gradient simultaneously.
The latter options allows the sharing of inner computations.
That is, the user can provide the methods {\tt Evaluate()} and {\tt Gradient()},
or {\tt EvaluateWithGradient()}.
For the example objective function above,
we empirically show (Section~\ref{sec:experiments}) that the ability to provide
{\tt EvaluateWithGradient()} can result in a significant speedup.

Similarly, when implementing a differentiable optimizer in {\tt ensmallen},
it is possible to use {\it either} {\tt Evaluate()} and {\tt Gradient()},
{\it or} {\tt EvaluateWithGradient()} during optimization.

The same technique can be used to infer and provide more missing methods than
just {\tt EvaluateWithGradient()} or {\tt Evaluate()} and {\tt Gradient()}.  
For instance, separable functions, differentiable separable functions, constrained
functions, and categorical functions each have inferrable methods.  Not all of
these possibilities are currently implemented in {\tt ensmallen}, but the
existing framework makes it straightforward to add more.  Below are a
few examples that {\tt ensmallen} currently implements:

\begin{itemize}
  \item {\it (Differentiable functions.)}  If the user provides an objective
function with {\tt Evaluate()} and {\tt Gradient()}, we can automatically
synthesize {\tt EvaluateWithGradient()}.

  \item {\it (Differentiable functions.)}  If the user provides an objective
function with {\tt EvaluateWithGradient()}, we can synthesize {\tt Evaluate()}
and/or {\tt Gradient()}.

  \item {\it (Separable functions.)}  If the user provides an objective
function with {\tt Evaluate()} and {\tt NumFunctions()}, we can sythesize a
non-separable version of {\tt Evaluate()}.

  \item {\it (Separable functions.)}  If the user provides an objective function
with {\tt Gradient()} and {\tt NumFunctions()}, we can synthesize a
non-separable version of {\tt Gradient()}.

  \item {\it (Separable functions.)}  If the user provides an objective
function with {\tt EvaluateWithGradient()} and {\tt NumFunctions()}, we can
synthesize a non-separable version of {\tt Gradient()}.
\end{itemize}

For more precise details on exactly how the method generation works,
see Appendix~\ref{sec:automatic_details},
where we describe the framework in a simplified form,
focusing only the {\tt EvaluateWithGradient()}/{\tt Evaluate()}/{\tt Gradient()}
example described above.
