\section{Overview}
\label{sec:overview}

% The primary goal of the {\tt ensmallen} library is to provide an easy-to-use
% and efficient facilities that can solve the problem $\operatorname{argmin}_x f(x)$
% for any function $f(x)$ that takes a vector or matrix input $x$.  
% To accomplish this goal, the library provides a large {set of optimizers} for optimizing
% {user-defined objective functions} in C++.  These optimizers are generic and
% flexible, meaning that they can support a wide range of use cases and
% applications.  Table~\ref{tab:comparison} contrasts the functionality supported
% by {\tt ensmallen} and other optimization toolkits.

% Many of these optimizers operate on
% specific {\it classes} of objective functions---for instance, {\tt ensmallen}
% provides many optimizers that work on differentiable functions, and so a user
% must provide both~$f(x)$ and~$f'(x)$ in such cases.
% The library allows the user to select an arbitrary type for~$x$
% (e.g., dense floating-point matrix, sparse integer matrix, etc.).

% Furthermore, in certain cases
% {\tt ensmallen} is able to infer methods that the optimizer can use beyond
% what the user has provided.  This support is internally accomplished via template
% metaprogramming and comes without runtime costs---all of the relevant work is done
% automatically at compile time.
% 
% In addition to allowing users to easily define and optimize their own objective
% functions, {\tt ensmallen} also makes it easy to implement new optimizers, which
% can then be contributed upstream and incorporated into the library.

The task of optimizing an objective function with {\tt ensmallen} is
straightforward.  The class of objective function (e.g., arbitrary, constrained,
differentiable, etc.) defines the implementation requirements.
Each objective function type has a minimal set of methods that must be implemented.
Typically this is only between one and four methods.
As an example,
to optimize an objective function $f(x)$ that is differentiable,
implementations of $f(x)$ and $f'(x)$ are required.
One of the optimizers for differentiable functions,
such as L-BFGS~\cite{liu1989limited},
can then be immediately employed.

Whenever possible, {\tt ensmallen} will automatically infer methods that are
not provided.  For instance, given a separable objective function where an
implementation of $f_i(x)$ is provided (as well as the number of such separable
objectives), an implementation of $f(x)$ can be automatically inferred.
This is done at compile-time, and so there is no additional runtime overhead
compared to a manual implementation.
C++ template metaprogramming techniques~\cite{abrahams2004c++,alexandrescu2001modern,
veldhuizen1998c++} are internally used to produce efficient code during compilation.

Not every type of objective function can be used with every type of optimizer.
For instance, since L-BFGS is a differentiable optimizer,
it cannot be used with a non-differentiable object function type
(e.g. an arbitrary function).
When an optimizer is used with a user-provided objective function,
an internal mechanism automatically checks the requirements,
resulting in user-friendly error messages if any required methods are not detected.




\subsection{Types of Objective Functions}

In most cases, the objective function $f(x)$ has inherent attributes;
for example, $f(x)$ might be differentiable.
The internal framework in {\tt ensmallen} can optionally take advantage of such attributes.
In the example of a differentiable function $f(x)$,
the user can provide an implementation of the gradient $f'(x)$,
which in turn allows a first-order optimizer to be used.  This generally leads
to significant speedups when compared to using only $f(x)$.
To allow exploitation of such attributes, the optimizers are built to work with
many types of objective functions.  The classes of objective functions
are listed below.

\begin{itemize}
\item {\bf Arbitrary functions} ({\tt ArbitraryFunctionType}).
No assumptions are made on function $f(x)$ and only the objective
$f(x)$ can be computed for a given $x$.

\item {\bf Differentiable functions} ({\tt DifferentiableFunctionType}).
A differentiable function $f(x)$ is an arbitrary function whose gradient $f'(x)$
can be computed for a given $x$, in addition to the objective.

\item {\bf Partially differentiable functions} ({\tt
PartiallyDifferentiableFunctionType}).  A partially differentiable function
$f(x)$ is a differentiable function with the additional property that the
gradient $f'(x)$ can be decomposed along some basis $j$ such that $f_j'(x)$ is
sparse.  Often, this is used for coordinate descent algorithms (i.e., $f'(x)$
can be decomposed into $f_{1}'(x)$, $f_{2}'(x)$, etc.).

\item {\bf Arbitrary separable functions} ({\tt
ArbitrarySeparableFunctionType}).  An arbitrary separable function is an
arbitrary function $f(x)$ that can be decomposed into the sum of several
objective functions: $f(x) = \sum\nolimits_i f_i(x)$.

\item {\bf Differentiable separable functions} ({\tt
DifferentiableSeparableFunctionType}).  A differentiable separable function is a
separable arbitrary function $f(x)$ where the individual gradients $f_i'(x)$ are
also computable.

\item {\bf Categorical functions} ({\tt CategoricalFunctionType}).  A
categorical function type is an arbitrary function $f(x)$ where some (or all)
dimensions of $x$ take discrete values from a set.

\item {\bf Constrained functions} ({\tt ConstrainedFunctionType}).
A constrained function $f(x)$ is a differentiable function%
\footnote{Generally, constrained functions  do not need to be differentiable.
However, this is a requirement here, as all of the current optimizers in {\tt ensmallen}
for constrained functions require a gradient to be available.}
subject to constraints of the form $c_i(x)$; when the constraints are satisfied, $c_i(x) = 0\; \forall \; i$.
Minimizing $f(x)$ then means minimizing $f(x) + \sum_i c_i(x)$.

\item {\bf Semidefinite programs} (SDPs).  {\it (These are a subset of
constrained functions.)}  {\tt ensmallen} has special
support to make optimizing semidefinite
programs~\cite{vandenberghe1996semidefinite} straightforward.
\end{itemize}

For details on the required signatures for each objective function type
(such as {\tt DifferentiableFunctionType}),
see the online documentation at \mbox{\url{https://ensmallen.org/docs.html}}.


\subsection{Pre-Built Optimizers}

For each class of the objective functions,
{\tt ensmallen} provides a set of pre-built optimizers:

\begin{itemize}
  \item {\bf For arbitrary functions:}  Simulated
annealing~\cite{kirkpatrick1983optimization}, CNE
(Conventional Neural Evolution)~\cite{montana1989training}, DE (Differential
Evolution)~\cite{storn1997differential}, PSO (Particle Swarm
Optimization)~\cite{Kennedy1995}, SPSA (Simultaneous Perturbation
Stochastic Approximation)~\cite{spall1992multivariate}.

  \item {\bf For differentiable functions:}  L-BFGS~\cite{liu1989limited},
Frank-Wolfe~\cite{jaggi2013revisiting}, gradient descent.

  \item {\bf For partially differentiable functions.}  SCD (Stochastic
Coordinate Descent)~\cite{Shalev-Shwartz2009}.

  \item {\bf For arbitrary separable functions:}  CMA-ES (Covariance Matrix
Adaptation Evolution Strategy)~\cite{Hansen2001}.

  \item {\bf For differentiable separable functions:}
AdaBound~\cite{Luo2019AdaBound},
AdaDelta~\cite{zeiler2012adadelta}, AdaGrad~\cite{duchi2011adaptive},
Adam~\cite{Kingma2014}, AdaMax~\cite{Kingma2014},
AMSBound~\cite{Luo2019AdaBound}, AMSGrad~\cite{reddi2019convergence},
Big Batch SGD~\cite{De2017}, Eve~\cite{Koushik2016}, FTML (Follow The Moving
Leader)~\cite{Zheng2017},
Hogwild!~\cite{recht2011hogwild}, IQN
(Incremental Quasi-Newton)~\cite{1106.5730}, Katyusha~\cite{Allen-Zhu2016},
Lookahead~\cite{Zhang2019}, SGD with momentum~\cite{rumelhart1988learning},
Nadam~\cite{Dozat2015},
NadaMax~\cite{Dozat2015}, SGD with Nesterov momentum~\cite{Nesterov1983},
Optimistic
Adam~\cite{daskalakis2017training}, QHAdam (Quasi-Hyperbolic
Adam)~\cite{ma2019qh}, QHSGD
(Quasi-Hyperbolic Stochastic Gradient Descent)~\cite{ma2019qh},
RMSProp~\cite{tieleman2012lecture},
SARAH/SARAH+~\cite{Nguyen2017}, stochastic gradient descent, SGDR (Stochastic Gradient
Descent with Restarts)~\cite{Loshchilov2016}, Snapshot SGDR~\cite{Huang2017},
SMORMS3~\cite{Funk2015}, SVRG (Stochastic Variance Reduced
Gradient)~\cite{Johnson2013}, SWATS~\cite{Keskar2017},
SPALeRA (Safe Parameter-wise Agnostic LEarning Rate
Adaptation)~\cite{Schoenauer2017},
WNGrad~\cite{Wu2018}.

  \item {\bf For categorical functions:}  Grid search.

  \item {\bf For constrained functions:}  Augmented Lagrangian method,
primal-dual interior point SDP solver, LRSDP (low-rank accelerated SDP
solver)~\cite{burer2003nonlinear}.
\end{itemize}


\subsection{Example Usage}
\label{sec:linreg_example}

Let us consider the problem of linear regression, where we are
given a matrix of predictors $\bm X \in \mathcal{R}^{n \times d}$ and a vector
of responses $\bm y \in \mathcal{R}^n$.  Our task is to find the best linear
model $\bm \theta \in \mathcal{R}^d$; that is, we want to find
$\bm \theta^* = \operatornamewithlimits{argmin}_{\bm\theta} f(\bm \theta)$ for
%
\begin{equation}
f(\bm \theta) = \| \bm X \bm \theta - \bm y \|^2 = (\bm X \bm \theta - \bm y)^T
(\bm X \bm \theta - \bm y).
\label{eqn:obj_lr}
\end{equation}

\noindent
From this we can derive the gradient $f'(\bm \theta)$:
%
\begin{equation}
f'(\bm \theta) = 2 \bm X^T (\bm X \bm \theta - \bm y).
\label{eqn:grad_lr}
\end{equation}

To find $\bm \theta^*$ using a differentiable
optimizer, we simply need to provide implementations of $f(\bm \theta)$ and
$f'(\bm \theta)$ according to the signatures required by the {\tt
DifferentiableFunctionType} of objective function.  For a differentiable
function, only two methods are necessary: {\tt Evaluate()} and {\tt Gradient()}.
The pre-built L-BFGS optimizer can be used to find $\bm \theta^*$.
% we just need to provide an implementation of $f(\bm \theta)$ and $f'(\bm \theta)$
% as L-BFGS requires a differentiable objective function.

Figure~\ref{fig:lr_function} shows an example implementation.
We hold {\tt X} and {\tt y} as members of the
{\tt LinearRegressionFunction class},
and {\tt theta} is used to represent $\bm \theta$.
Via the use of Armadillo~\cite{sanderson2016armadillo},
the linear algebra expressions to implement the objective function and gradient
are readable in a way that closely matches Equations~(\ref{eqn:obj_lr}) and~(\ref{eqn:grad_lr}).


% Details on how to implement and use each type of objective function 
% are omitted here for brevity.
% They can be found in the online documentation for {\tt ensmallen}
% at \mbox{\url{https://ensmallen.org/docs.html}}.
%%The details are subject to evolution over time.



\begin{figure}[!b]
\hrule
\vspace{1ex}
\centering
\begin{minted}[fontsize=\small]{c++}
#include <ensmallen.hpp>

class LinearRegressionFunction
{
 public:
  LinearRegressionFunction(const arma::mat& in_X, const arma::vec& in_y) : X(in_X), y(in_y)  { }

  double Evaluate(const arma::mat& theta)  { return (X * theta - y).t() * (X * theta - y); }

  void Gradient(const arma::mat& theta, arma::mat& gradient)  { gradient = 2 * X.t() * (X * theta - y); }

 private:
  const arma::mat& X;
  const arma::vec& y;
};

int main()
{
  arma::mat X;
  arma::vec y;
  
  // ... set the contents of X and y here ...
  
  ens::LinearRegressionFunction f(X, y);

  ens::L_BFGS optimizer; // create the optimizer with default parameters

  arma::mat theta_best(X.n_rows, 1, arma::fill:randu);  // initial starting point (uniform random values)

  optimizer.Optimize(f, theta_best);
  // at this point theta_best contains the best parameters

  return 0;
}
\end{minted}
\hrule
\vspace*{-0.5em}
\caption{An example implementation of an objective function class for linear
regression and usage of the L-BFGS optimizer in {\tt ensmallen}.
The online documentation for all ensmallen optimizers
is at \mbox{\url{https://ensmallen.org/docs.html}}.
The {\tt arma::mat} and {arma::vec} types are 
dense matrix and vector classes
from the Armadillo linear algebra library~\cite{sanderson2016armadillo},
with the corresponding online documentation at \mbox{\url{http://arma.sf.net/docs.html}}.
}
\label{fig:lr_function}
\end{figure}
