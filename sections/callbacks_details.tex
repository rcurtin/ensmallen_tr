\section{Callbacks -- Details}
\label{sec:callback_details}

In Section~\ref{sec:callbacks}, we introduced {\tt ensmallen}'s support for
callbacks.  This support is implemented via templates---like much of the rest of
{\tt ensmallen}.  This design strategy was chosen in order to minimize runtime
overhead caused by callbacks, and produce machine code roughly equivalent to the
machine code that would have been produced if the callback had been directly
integrated into the optimizer's code.

In this section, we detail the implementation of these callbacks and show how to
implement optimizers that are capable of handling these callbacks.

An optimizer that supports callbacks should call out to the static methods in
the {\tt Callback} class inside of its {\tt Optimize()} method, as below:
%
\begin{minted}[fontsize=\small]{c++}
    double functionValue = function.Evaluate(coordinates);
    bool terminate = Callback::Evaluate(*this, function, coordinates, functionValue, callbacks...);
\end{minted}

In the above snippet, {\tt function} represents the function being optimized,
{\tt *this} is the optimizer itself, {\tt coordinates} represents the current
coordinates of the optimization, and {\tt callbacks} is a template vararg pack
containing all of the given callbacks.
Note that this means {\tt callbacks} can be empty.

The {\tt Callback} class contains a top-level method to use for each of the
callback types supported by {\tt ensmallen}, as listed in Table~\ref{tab:callback_list}.
There is a {\tt Callback::Gradient()} method, {\tt Callback::EvaluateConstraint()},
{\tt Callback::BeginEpoch()}, and so forth. 

\begin{figure}[b!]
\hrule
\vspace{1ex}
\begin{minted}[fontsize=\small]{c++}
template<typename OptimizerType, typename FunctionType, typename MatType, typename... CallbackTypes>
static bool Callback::Evaluate(OptimizerType& optimizer,
                               FunctionType& function,
                               const MatType& coordinates,
                               const double objective,
                               CallbackTypes&... callbacks)
{
  // This will return immediately once a callback returns true.
  bool result = false;
  (void) std::initializer_list<bool>{ result =
      result || Callback::EvaluateFunction(callbacks, optimizer, function,
      coordinates, objective)... };
   return result;
}
\end{minted}
\hrule
\vspace*{-0.5em}
\caption{Implementation of {\tt Callback::Evaluate()},
showing part of the process of calling each callback
given in the template vararg pack {\tt callbacks}.
\TODO{possible inconsistency/confusion with Fig.~\ref{fig:example_prog_callbacks_2}
  which returns {\tt false} to indicate failure;
  if this isn't an inconsistency, need to briefly explain why}
}
\label{fig:callback_evaluate}
\end{figure}

The {\tt Callback::Evaluate()} method calls the {\tt Evaluate()} method of every
given callback that has an {\tt Evaluate()} method implemented.  This means that
there is some amount of difficulty we have to handle: not every callback in {\tt
callbacks...} will have an {\tt Evaluate()} method available.  Thus, we can use
the same techniques as in Section~\ref{sec:automatic} to detect, via SFINAE,
whether an {\tt Evaluate()} method exists for a given callback, and then perform
the correct action based on the result.

The definition of {\tt Callback::Evaluate()} is given in
Figure~\ref{fig:callback_evaluate}.  Each callback function takes several
template parameters, including {\tt OptimizerType} (the type of optimizer that
is being used), {\tt FunctionType} (the type of function being optimized), {\tt
MatType} (the matrix type used to store the coordinates), and {\tt
CallbackTypes} (the set of types of callbacks).  Of these, {\tt CallbackTypes}
is the most important.  Since it is a template vararg, the
type {\tt CallbackTypes} is actually a pack that corresponds to every type of
every callback that must be called.

The implementation of the function unpacks the {\tt callbacks},
calling each callback's {\tt Evaluate()} method (if it exists) in sequence,
and terminating early if any of these callbacks returns {\tt true}.
\TODO{should this be ``false'' to indicate optimization failure?
  possible inconsistency/confusion with Fig.~\ref{fig:example_prog_callbacks_2};
  if this isn't an inconsistency, need to briefly explain why}
Each callback's {\tt Evaluate()} method is called through the helper
function {\tt Callback::EvaluateFunction()},
which uses SFINAE traits to control behavior depending on whether
the given callback has an {\tt Evaluate()} method or not.
Figure~\ref{fig:callback_evaluate_function} shows the two overloads of
{\tt Callback::EvaluateFunction()}.

\begin{figure}[t!]
\hrule
\vspace{1ex}
\begin{minted}[fontsize=\small]{c++}
template<typename CallbackType, typename OptimizerType, typename FunctionType, typename MatType>
static typename std::enable_if<callbacks::traits::HasEvaluateSignature<
    CallbackType, OptimizerType, FunctionType, MatType>::value, bool>::type
EvaluateFunction(CallbackType& callback,
                 OptimizerType& optimizer,
                 FunctionType& function,
                 const MatType& coordinates,
                 const double objective)
{
  return (const_cast<CallbackType&>(callback).Evaluate(optimizer, function, coordinates, objective),
      false);
}

template<typename CallbackType, typename OptimizerType, typename FunctionType, typename MatType>
static typename std::enable_if<!callbacks::traits::HasEvaluateSignature<
    CallbackType, OptimizerType, FunctionType, MatType>::value, bool>::type
EvaluateFunction(CallbackType& /* callback */,
                 OptimizerType& /* optimizer */,
                 FunctionType& /* function */,
                 const MatType& /* coordinates */,
                 const double /* objective */)
{ return false; }
\end{minted}
\hrule
\vspace*{-0.5em}
\label{fig:callback_evaluate_function}
\caption{Implementation of {\tt Callback::EvaluateFunction()}.  Two overloads
are specified: the first is used when {\tt CallbackType} has an {\tt Evaluate()}
method, and the second is used when {\tt CallbackType} does not have an {\tt
Evaluate()} method.  SFINAE is used, via {\tt
callbacks::traits::HasEvaluateSignature}, to determine which overload to use.}
\end{figure}

Similar to Section~\ref{sec:automatic}, a traits class is used to determine
which of the two overloads of {\tt EvaluateFunction()} to call.  The traits
class {\tt callbacks::traits::HasEvaluateSignature<...>::value}
evaluates to {\tt true} only when the inner callback and arguments can form a
valid {\tt Evaluate()} signature.  Thus, the correct overload is called when
{\tt EvaluateFunction()} is called from {\tt Callback::Evaluate()}.

Each of the other callbacks in the {\tt Callback} class is implemented in
virtually identical form, although with different arguments depending on what
the callback is.  For further details on the callback infrastructure,
examples are provided in the code repository in the directory
{\tt include/ensmallen\_bits/callbacks/}.

%For the sake of brevity we omit the expressions for the {\tt BeginOptimization}
%method \footnote{Interested readers can find that code in {\tt
%ensmallen\_bits/callbacks/callbacks.hpp}.}. The function depend heavily on
%SFINAE techniques for method detection; {\tt HasBeginOptimization} will evaluate
%to {\tt true} if {\tt CallbackType} has {\tt BeginOptimization()} and {\tt
%false} otherwise.

%Using the boolean template variable, we can then use template specialization to
%control whether {\tt BeginOptimization} returns a boolean, void to control the
%optimization process or if the callback ends in an empty function
%call.











%A custom callback is a powerful tool to customize the behavior of a function
%during the optimization process. Examples include {\tt StoreBestCoordinates<>()}
%where the model is automatically saved during the optimization process or {\tt
%EarlyStoppingAtMinLoss()}  which stops the optimization process when the minimum
%of loss has been reached.

% RC: seems like we already have a similar example, so I've commented this one
% out.
%\begin{figure}[t!]
%\centering
%\hrule
%\vspace{1ex}
%\begin{adjustbox}{minipage=\columnwidth,scale={0.90}{0.85}}
%\begin{minted}[fontsize=\small]{c++}
%class EarlyStoppingAtMinLoss
%{
% public:
%  // Set up the early stopping at min loss class, which keeps track of the
%  // minimum loss.
%  EarlyStop() : bestObjective(std::numeric_limits<double>::max()) { }
%
%  // Callback function called at the end of a pass over the data, which provides
%  // the current objective. We are only interested in the objective and ignore
%  // the rest.
%  template<typename OptimizerType, typename FunctionType, typename MatType>
%  void EndEpoch(OptimizerType&, FunctionType&, const MatType&, const size_t,
%                const double objective)
%  {
%    // Check if the given objective is lower as the previous objective.
%    if (objective < bestObjective)
%      bestObjective = objective; // Update the local objective.
%    else
%      return true; // Stop the optimization process.
%
%    return false; // Do not stop the optimization process.
%  }
%
%  double bestObjective;
%};
%\end{minted}


%\begin{minted}[fontsize=\small]{c++}
%// We assume that "X" and "y" are given.
%LinearRegressionFunction f(X, y);
%
%L_BFGS optimizer; // Create the optimizer with all default parameters.
%
%// The theta_best matrix will hold the best model that we find after we call
%// Optimize(); for now, we set it to the initial point (uniform random values).
%arma::mat theta_best(X.n_rows, 1, arma::fill:randu);
%
%// Optimize the given function and provide a callback (EarlyStoppingAtMinLoss)
%// that is called at the end of an epoch.
%optimizer.Optimize(f, theta_best, EarlyStoppingAtMinLoss());
%\end{minted}
%\end{adjustbox}
%\vspace{1ex}
%\hrule
%\caption
%  {
%  A simple C++ program that demonstrates the implementation of a custom callback function.
%  }
%\label{fig:example_prog_callbacks_parameter}
%\end{figure}
