\section{Static assertions to check matrix traits}
\label{sec:templated_optimize_details}

In Section~\ref{sec:templated_optimize}, we showed that {\tt ensmallen} uses
templates to allow arbitrary types to be used for the matrix type or gradient
type.  However, this requires some amount of special handling to ensure that the
user encounters a reasonable, comprehensible error message if the given matrix
or gradient type does not have the required methods.  Here, we discuss the
details of how we provide better error messages than the default onslaught of
compiler error messages via the use of {\tt static\_assert()}.

{\tt ensmallen} provides a few utility methods for checking traits of {\tt
MatType} and {\tt GradType} (for differentiable objective functions) inside of
an optimizer's {\tt Optimize()} method:

\begin{itemize}
  \item {\tt RequireFloatingPointType<MatType>()}: this requires that the
element type held by {\tt MatType} is either {\tt float} or {\tt double}.
This is generally needed by optimizers that use LAPACK or BLAS functionality.

  \item {\tt RequireSameInternalTypes<Mat1Type, Mat2Type>()}: this requires that
{\tt Mat1Type} and {\tt Mat2Type} use the same type to hold elements.  This can
be useful for differentiable optimizers, where we want to ensure, for example,
that a user can't pass a {\tt MatType} that holds {\tt int}s, but a {\tt GradType} that
holds {\tt unsigned int}s.

  \item {\tt RequireDenseFloatingPointType<MatType>()}: this requires that the
element type held by {\tt MatType} is either {\tt float} or {\tt double}, and
that the representation used for storage by {\tt MatType} is dense.
\end{itemize}

These methods would typically be called at the top of the implementation of an
optimizer's {\tt Optimize()} method.

At the time of writing, these are the only three checks that have
been needed, but it is easy to add more.  The implementation of these functions
is just a {\tt static\_assert()} that uses some underlying traits of the given
template parameters.  For example, Figure~\ref{fig:rsit} is the implementation
of {\tt RequireSameInternalTypes<...>()}.

\begin{figure}[t]
\hrule
\vspace{1ex}
\begin{minted}[fontsize=\small]{c++}
/**
 * Require that the internal element type of the matrix type and gradient type
 * are the same.  A static_assert() will fail if not.
 */
template<typename MatType, typename GradType>
void RequireSameInternalTypes()
{
#ifndef ENS_DISABLE_TYPE_CHECKS
  static_assert(std::is_same<typename MatType::elem_type,
                             typename GradType::elem_type>::value,
      "The internal element types of the given MatType and GradType must be "
      "identical, or it is not known to work!  If you would like to try "
      "anyway, set the preprocessor macro ENS_DISABLE_TYPE_CHECKS before "
      "including ensmallen.hpp.  However, you get to pick up all the pieces if "
      "there is a failure!");
#endif
}
\end{minted}
\hrule
\vspace*{-0.5em}
\caption{Implementation of {\tt RequireSameInternalTypes()} from internal
{\tt ensmallen} code.
\TODO{This figure is probably not be necessary: it's trivial stuff. The
descriptions in the preceding bullet points are sufficient.  (Response from
Ryan: I agree, but I find that people new to mlpack are really scared of
actually *looking* at code and figuring out what it does.  This gives a direct
reference to code that's there, which is an entrypoint someone can use to then
learn more.  Anyway, it's in the back now, so maybe it's not a problem to leave
it in.  Let me know what you think.)}
}
\label{fig:rsit}
\end{figure}
