\section{Static Assertions to Check Matrix Traits}
\label{sec:templated_optimize_details}

In Section~\ref{sec:templated_optimize}, we showed that {\tt ensmallen} uses
templates to allow arbitrary types to be used for the matrix type or gradient
type.
However, care must be taken there to avoid a drawback with C++ template metaprogramming,
which is the issue of compiler errors.
In Figure~\ref{fig:gd}, the compiler may fail to substitute a given {\tt MatType}
into the code for {\tt GradientDescent::Optimize()}. A~typical reason might be that
a required method of {\tt MatType} or {\tt FunctionType} is not available.
For example, the given {\tt MatType} does not have an {\tt operator\*()} method.
Ordinarily, current C++ compilers typically emit a long list of error messages,
with unnecessarily detailed and distracting information.
The internal framework in {\tt ensmallen} avoids this issue via the use of
C++ compile-time static assertions ({\tt static\_assert()}),
resulting in considerably clearer error messages.

The framework provides a set of utility methods for checking traits of {\tt
MatType} and {\tt GradType} (for differentiable objective functions) inside of
an optimizer's {\tt Optimize()} method:

\begin{itemize}
  \item {\tt RequireFloatingPointType<MatType>()}: this requires that the
element type held by {\tt MatType} is either {\tt float} or {\tt double}.
This is generally needed by optimizers that use Armadillo functionality which
ends up calling LAPACK functions~\cite{TODO} \TODO{citation to LAPACK manual}.

  \item {\tt RequireSameInternalTypes<Mat1Type, Mat2Type>()}: this requires that
{\tt Mat1Type} and {\tt Mat2Type} use the same type to hold elements.  This can
be useful for differentiable optimizers, where we want to ensure, for example,
that a user cannot pass a {\tt MatType} that holds {\tt int}s, but a {\tt GradType} that
holds {\tt unsigned int}s.
\TODO{unclear or poor wording}

  \item {\tt RequireDenseFloatingPointType<MatType>()}: this requires that the
element type held by {\tt MatType} is either {\tt float} or {\tt double}, and
that the representation used for storage by {\tt MatType} is dense.
\end{itemize}

These methods would typically be called at the top of the implementation of an
optimizer's {\tt Optimize()} method.

At the time of writing, these are the only three checks that have
been needed, but it is easy to add more.  The implementation of these functions
is just a {\tt static\_assert()} that uses some underlying traits of the given
template parameters.  For example, Figure~\ref{fig:rsit} is the implementation
of {\tt RequireSameInternalTypes<...>()}.

Note that users who would like to continue despite these {\tt static\_assert()}
checks failing may simply define the macro {\tt ENS\_DISABLE\_TYPE\_CHECKS} in
their code before the line {\tt \#include <ensmallen.hpp>}.


\begin{figure}[t]
\hrule
\vspace{1ex}
\begin{minted}[fontsize=\small]{c++}
/**
 * Require that the internal element type of the matrix type and gradient type
 * are the same.  A static_assert() will fail if not.
 */
template<typename MatType, typename GradType>
void RequireSameInternalTypes()
{
#ifndef ENS_DISABLE_TYPE_CHECKS
  static_assert(std::is_same<typename MatType::elem_type,
                             typename GradType::elem_type>::value,
      "The internal element types of the given MatType and GradType must be "
      "identical, or it is not known to work!  If you would like to try "
      "anyway, set the preprocessor macro ENS_DISABLE_TYPE_CHECKS before "
      "including ensmallen.hpp.  However, you get to pick up all the pieces if "
      "there is a failure!");
#endif
}
\end{minted}
\hrule
\vspace*{-0.5em}
\caption{Implementation of {\tt RequireSameInternalTypes()} from internal
{\tt ensmallen} code.
}
\label{fig:rsit}
\end{figure}
